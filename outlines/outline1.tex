%Carthage Physics Senior Thesis
%Author: Aaron Scheets

\documentclass[twocolumn,12pth]{article}
\usepackage[hmargin={1.25in,0.75in},bmargin=1in,tmargin=1in]{geometry} 
\usepackage{setspace}
\usepackage{url}
\usepackage{syntonly}
\usepackage[ampersand]{easylist}

\usepackage{natbib}

\usepackage{amsmath}
\usepackage{mathtools}
\usepackage{gensymb}
\usepackage{amssymb}
\usepackage{float}
\providecommand{\e}[1]{\ensuremath{\times 10^{#1}}}

\usepackage{graphicx}
\usepackage{caption}


\title{Writing A Fluid Solver From First Principles-Outline}

\author{Aaron Scheets}

\onecolumn

\begin{document}

\maketitle

\section{Introduction}

\begin{easylist}[enumerate]
& Motivation/hook
& History/context
& Overview sort of...

\end{easylist}

\section{Components of a Numerical Solution Method}

\begin{easylist}[enumerate]

& Mathematical Model
&& First of all, what are you modeling?
&&& Incompressible fluid flow in a channel
&&& What are the physical implications of the fluid adjectives used here?
&& Why are you modeling this?
&& How are you modeling it, what set of equations are you using?
&& Why are you modeling it this way?
& Discretization Method
&& I am using finite differences to discretize the equations
&& What does it mean to ``discretize''?
&& How do finite differences discretize the equations in question?
&& How well do finite differences match the physics in question?
& Coordinate and Basis Vector System
&& Working in Cartesian Coordinates
&& Why are you working in cartesian coordinates?
&& What would change if using different coordinate system?
&& ``Basis in which vectors and tensors will be defined''?
& Numerical Grid
& Finite Approximations
& Solution Method
& Convergence Criteria

\end{easylist}

\end{document}

