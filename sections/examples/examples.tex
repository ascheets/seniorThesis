%Carthage Physics Senior Thesis
%Author: Aaron Scheets

\documentclass[twocolumn,12pth]{article}
\usepackage[hmargin={1.25in,0.75in},bmargin=1in,tmargin=1in]{geometry} 
\usepackage{setspace}
\usepackage{url}
\usepackage{syntonly}
\usepackage[ampersand]{easylist}

\usepackage{natbib}

\usepackage{amsmath}
\usepackage{mathtools}
\usepackage{gensymb}
\usepackage{amssymb}
\usepackage{float}
\providecommand{\e}[1]{\ensuremath{\times 10^{#1}}}

\usepackage{graphicx}
\usepackage{caption}


\title{Writing A Fluid Solver From First}

\author{Aaron Scheets}

\onecolumn

\begin{document}

\maketitle

\section{Discretizing the Navier-Stokes Equations}

The fluid flow simulation is 2-dimensional, thus, we only need to consider two of the component equations from the Navier-Stokes Equations:

\begin{equation}
\frac{\partial{u}}{\partial{t}} + u\frac{\partial{u}}{\partial{x}} + v\frac{\partial{u}}{\partial{y}} = g_x - \frac{1}{\rho}\frac{\partial{p}}{\partial{x}} + \nu \bigg( \frac{\partial^2u}{\partial{x}^2} + \frac{\partial^2u}{\partial{y}^2} \bigg)
\label{eq:exs1}
\end{equation}

\begin{equation*}
\frac{\partial{v}}{\partial{t}} + u\frac{\partial{v}}{\partial{x}} + v\frac{\partial{v}}{\partial{y}} = g_y -\frac{1}{\rho}\frac{\partial{p}}{\partial{y}} + \nu \bigg( \frac{\partial^2v}{\partial{x}^2} + \frac{\partial^2v}{\partial{y}^2} \bigg)
\end{equation*}

The third equation we will need to discretize is the Poisson pressure equation which ensures the divergence of the velocity will be zero.

\begin{equation}
\frac{\partial^2p}{\partial{x}^2} + \frac{\partial^2p}{\partial{y}^2} = \rho \bigg[ \frac{1}{\Delta{t}}\bigg(\frac{\partial{u}}{\partial{x}} + \frac{\partial{v}}{\partial{y}} \bigg) - \bigg(\frac{\partial{u}}{\partial{x}} \bigg)^2 - 2\frac{\partial{v}}{\partial{x}}\frac{\partial{u}}{\partial{y}} - \bigg(\frac{\partial{v}}{\partial{y}}  \bigg)^2  \bigg]
\label{eq:exs2}
\end{equation}

\subsection{Unsteady Term}

The notation used when discretizing continous equations is key to identifying the type of derivative being taken, and the numerical scheme being used to take that derivative.
To demonstrate the discretization notation, consider the unsteady term of the component equations.
The unsteady term in the Navier-Stokes Equations is $\frac{\partial{\mathbf{v}}}{\partial{t}}$, which corresponds to $\frac{\partial{u}}{\partial{t}}$ and $\frac{\partial{v}}{\partial{t}}$ in the component equations above.
It is useful to discretize the unsteady term using a forward difference.

\begin{equation}
\frac{\partial{u}}{\partial{t}} \approx \frac{u_{i,j}^{n+1} - u_{i,j}^{n}}{\Delta{t}}
\end{equation}

The $u$ terms in the expression above represent the x-component of velocity.
The subscripts $i$ and $j$, indicate the position on the numerical grid where the velocity is being calculated.
The superscript $n$ indicates the moment in time in which we are considering the velocity.
One should notice that the indices indicating position, $i,j$ are constant, in contrast to the index indicating time, $n$.
This is a characteristic of the partial derivative with respect to time; time evolves while the other variables are held constant.
By discretizing the unsteady term using a forward difference, the velocity one step forward in time, $u_{i}^{n+1}$, is accessible.
Thus, solving for $u_{i}^{n+1}$, stepping the velocity field forward in time is possible.

\subsection{Convective Terms}

The convection terms from the component equations are the following:

\begin{equation}
u\frac{\partial{u}}{\partial{x}} + v\frac{\partial{u}}{\partial{y}}
\end{equation}

\begin{equation*}
u\frac{\partial{v}}{\partial{x}} + v\frac{\partial{v}}{\partial{y}}
\end{equation*}

The convective terms are discretized using a backward difference approach.
Explain why this is the case?
The discretized form of the convective components is the following:

\begin{equation}
u_{i,j}^n\frac{u^n_{i,j} - u^n_{i-1,j}}{\Delta{x}} + v_{i,j}^n\frac{u^n_{i,j} - u^n_{i,j-1}}{\Delta{y}}
\end{equation}

\begin{equation*}
u_{i,j}^n\frac{v^n_{i,j} - v^n_{i-1,j}}{\Delta{x}} + v_{i,j}^n\frac{v^n_{i,j} - v^n_{i,j-1}}{\Delta{y}}
\end{equation*}

The partial derivatives with respect to x have changing $i$ indices and are constant in $j$.
The partials with respect to y are just the opposite, constant in $i$ and variable in $j$.

\subsection{Pressure Terms}

The pressure terms from the component equations are the following:

\begin{equation}
\frac{\partial{p}}{\partial{x}}
\end{equation}

\begin{equation*}
\frac{\partial{p}}{\partial{y}}
\end{equation*}

The partial derivatives of pressure with respect to x and y are approximated using a central difference approximation.
Explain why this is the case?
The discretized form of the pressure terms are the following:

\begin{equation}
\frac{p^n_{i+1,j} - p^n_{i-1,j}}{\Delta{2x}} 
\end{equation}

\begin{equation*}
\frac{p^n_{i,j+1} - p^n_{i,j-1}}{\Delta{2y}}
\end{equation*}

\subsection{Diffusion Terms}

The diffusive, viscous terms from the component equations are the following:

\begin{equation}
\nu \bigg( \frac{\partial^2u}{\partial{x}^2} + \frac{\partial^2u}{\partial{y}^2} \bigg)
\end{equation}

\begin{equation*}
\nu \bigg( \frac{\partial^2v}{\partial{x}^2} + \frac{\partial^2v}{\partial{y}^2} \bigg)
\end{equation*}

The diffusive terms are discretized using a second order, central difference derivative approximation.
Explain why this is the case?
The discretized form of the diffusive terms is the following:

\begin{equation}
\nu \Bigg( \frac{u^n_{i+1,j} -2u^n_{i,j} +  u^n_{i-1,j}}{\Delta{x^2}} + \frac{u^n_{i,j+1} -2u^n_{i,j} + u^n_{i,j-1}}{\Delta{y^2}} \Bigg)
\end{equation}

\begin{equation*}
\nu \Bigg( \frac{v^n_{i+1,j} -2v^n_{i,j} + v^n_{i-1,j}}{\Delta{x^2}} + \frac{v^n_{i,j+1} -2v^n_{i,j} + v^n_{i,j-1}}{\Delta{y^2}} \Bigg)
\end{equation*}

\subsection{Poisson Equation}

The left hand side of the Poisson pressure equation, (\ref{eq:exs2}) consists of two second order, spatial derivatives.
These derivatives are discretized using central, second order finite difference approximations:

\begin{equation}
\frac{\partial^2p}{\partial{x}^2} \approx \Bigg( \frac{p^n_{i+1,j} -2p^n_{i,j} + p^n_{i-1,j}}{\Delta{x^2}} \Bigg)
\end{equation}

\begin{equation*}
\frac{\partial^2p}{\partial{y}^2} \approx \Bigg( \frac{p^n_{i,j+1} -2p^n_{i,j} + p^n_{i,j-1}}{\Delta{y^2}} \Bigg)
\end{equation*}

The right hand side of the pressure equation consists of first order, spatial derivatives of the velocity components.
These derivatives are all approximated uisng central, first order finite differences.
The pattern in describing finite differences should be apparent at this point.
Including the discretized version of the right hand side of the pressure equation here would be more redundant than helpful.

\subsection{Solving for Forward Step in Time}



\bibliographystyle{plainnat}
\nocite{*}
\bibliography{~/Documents/Carthage/seniorthesis/writing/seniorThesis/library.bib}

\end{document}

