\subsection{Discretizing the Navier-Stokes Equations}

With all of the different types of finite differences at our disposal we can now come up with an approximation to the Navier-Stokes Equations.
In the last sections, forward, backward, and central finite differences were presented for first and second order derivatives.
Our goal is to approximate the component equations of the Navier-Stokes vector equation using those finite differences.
The fluid flow simulation is 2-dimensional, thus, we only need to consider two of the component equations from the Navier-Stokes Equations:

\begin{equation}
\frac{\partial{u}}{\partial{t}} + u\frac{\partial{u}}{\partial{x}} + v\frac{\partial{u}}{\partial{y}} = g_x - \frac{1}{\rho}\frac{\partial{p}}{\partial{x}} + \nu \bigg( \frac{\partial^2u}{\partial{x}^2} + \frac{\partial^2u}{\partial{y}^2} \bigg)
\label{eq:exs1}
\end{equation}

\begin{equation*}
\frac{\partial{v}}{\partial{t}} + u\frac{\partial{v}}{\partial{x}} + v\frac{\partial{v}}{\partial{y}} = g_y -\frac{1}{\rho}\frac{\partial{p}}{\partial{y}} + \nu \bigg( \frac{\partial^2v}{\partial{x}^2} + \frac{\partial^2v}{\partial{y}^2} \bigg)
\end{equation*}

Each derivative in the component equations could be discretized differently, that is, we have a choice as to whether we approximate the derivatives with forward, backward, or central differences.
The particular type of finite difference we decide to use for a particular derivative is known as the \textit{numerical scheme} for that derivative.
The derivatives of the component equations correspond to physical processes and the numerical scheme we choose will depend on the process in question.
In other words, some numerical schemes are better approximations to certain physical processes than others.
For example, the most accurate and stable approximation to the convection terms is a backwards finite difference.
Some analysis is required to show why this is the case, but the best finite difference option is a little more obvious for other terms.


The third equation we will need to discretize is the Poisson pressure equation which ensures the divergence of the velocity will be zero.

\begin{equation}
\frac{\partial^2p}{\partial{x}^2} + \frac{\partial^2p}{\partial{y}^2} = \rho \bigg[ \frac{1}{\Delta{t}}\bigg(\frac{\partial{u}}{\partial{x}} + \frac{\partial{v}}{\partial{y}} \bigg) - \bigg(\frac{\partial{u}}{\partial{x}} \bigg)^2 - 2\frac{\partial{v}}{\partial{x}}\frac{\partial{u}}{\partial{y}} - \bigg(\frac{\partial{v}}{\partial{y}}  \bigg)^2  \bigg]
\label{eq:exs2}
\end{equation}

\subsection{Unsteady Term}

The notation used when discretizing continous equations is key to identifying the type of derivative being taken, and the numerical scheme being used to take that derivative.
To demonstrate the discretization notation, consider the unsteady term of the component equations.
The unsteady term in the Navier-Stokes Equations is $\frac{\partial{\mathbf{v}}}{\partial{t}}$, which corresponds to $\frac{\partial{u}}{\partial{t}}$ and $\frac{\partial{v}}{\partial{t}}$ in the component equations above.
It is useful to discretize the unsteady term using a forward difference.

\begin{equation}
\frac{\partial{u}}{\partial{t}} \approx \frac{u_{i,j}^{n+1} - u_{i,j}^{n}}{\Delta{t}}
\end{equation}

The $u$ terms in the expression above represent the x-component of velocity.
The subscripts $i$ and $j$, indicate the position on the numerical grid where the velocity is being calculated.
The superscript $n$ indicates the moment in time in which we are considering the velocity.
One should notice that the indices indicating position, $i,j$ are constant, in contrast to the index indicating time, $n$.
This is a characteristic of the partial derivative with respect to time; time evolves while the other variables are held constant.
By discretizing the unsteady term using a forward difference, the velocity one step forward in time, $u_{i}^{n+1}$, is accessible.
Thus, solving for $u_{i}^{n+1}$, stepping the velocity field forward in time is possible.

\subsection{Convective Terms}

The convection terms from the component equations are the following:

\begin{equation}
u\frac{\partial{u}}{\partial{x}} + v\frac{\partial{u}}{\partial{y}}
\end{equation}

\begin{equation*}
u\frac{\partial{v}}{\partial{x}} + v\frac{\partial{v}}{\partial{y}}
\end{equation*}

The convective terms are discretized using a backward difference approach.
Explain why this is the case?
The discretized form of the convective components is the following:

\begin{equation}
u_{i,j}^n\frac{u^n_{i,j} - u^n_{i-1,j}}{\Delta{x}} + v_{i,j}^n\frac{u^n_{i,j} - u^n_{i,j-1}}{\Delta{y}}
\end{equation}

\begin{equation*}
u_{i,j}^n\frac{v^n_{i,j} - v^n_{i-1,j}}{\Delta{x}} + v_{i,j}^n\frac{v^n_{i,j} - v^n_{i,j-1}}{\Delta{y}}
\end{equation*}

The partial derivatives with respect to x have changing $i$ indices and are constant in $j$.
The partials with respect to y are just the opposite, constant in $i$ and variable in $j$.

\subsection{Pressure Terms}

The pressure terms from the component equations are the following:

\begin{equation}
\frac{\partial{p}}{\partial{x}}
\end{equation}

\begin{equation*}
\frac{\partial{p}}{\partial{y}}
\end{equation*}

The partial derivatives of pressure with respect to x and y are approximated using a central difference approximation.
Explain why this is the case?
The discretized form of the pressure terms are the following:

\begin{equation}
\frac{p^n_{i+1,j} - p^n_{i-1,j}}{\Delta{2x}} 
\end{equation}

\begin{equation*}
\frac{p^n_{i,j+1} - p^n_{i,j-1}}{\Delta{2y}}
\end{equation*}

\subsection{Diffusion Terms}

The diffusive, viscous terms from the component equations are the following:

\begin{equation}
\nu \bigg( \frac{\partial^2u}{\partial{x}^2} + \frac{\partial^2u}{\partial{y}^2} \bigg)
\end{equation}

\begin{equation*}
\nu \bigg( \frac{\partial^2v}{\partial{x}^2} + \frac{\partial^2v}{\partial{y}^2} \bigg)
\end{equation*}

The diffusive terms are discretized using a second order, central difference derivative approximation.
Explain why this is the case?
The discretized form of the diffusive terms is the following:

\begin{equation}
\nu \Bigg( \frac{u^n_{i+1,j} -2u^n_{i,j} +  u^n_{i-1,j}}{\Delta{x^2}} + \frac{u^n_{i,j+1} -2u^n_{i,j} + u^n_{i,j-1}}{\Delta{y^2}} \Bigg)
\end{equation}

\begin{equation*}
\nu \Bigg( \frac{v^n_{i+1,j} -2v^n_{i,j} + v^n_{i-1,j}}{\Delta{x^2}} + \frac{v^n_{i,j+1} -2v^n_{i,j} + v^n_{i,j-1}}{\Delta{y^2}} \Bigg)
\end{equation*}

\subsection{Poisson Equation}

The left hand side of the Poisson pressure equation, (\ref{eq:exs2}) consists of two second order, spatiaNNl derivatives.
These derivatives are discretized using central, second order finite difference approximations:

\begin{equation}
\frac{\partial^2p}{\partial{x}^2} \approx \Bigg( \frac{p^n_{i+1,j} -2p^n_{i,j} + p^n_{i-1,j}}{\Delta{x^2}} \Bigg)
\end{equation}

\begin{equation*}
\frac{\partial^2p}{\partial{y}^2} \approx \Bigg( \frac{p^n_{i,j+1} -2p^n_{i,j} + p^n_{i,j-1}}{\Delta{y^2}} \Bigg)
\end{equation*}

The right hand side of the pressure equation consists of first order, spatial derivatives of the velocity components.
These derivatives are all approximated uisng central, first order finite differences.
The pattern in describing finite differences should be apparent at this point.
Including the discretized version of the right hand side of the pressure equation here would be more redundant than helpful.

\subsection{Solving for Forward Step in Time}


The simulation is of fluid flow in a pipe, therefore the shape of the domain will be rectangular.
Let the domain span from $x = 0$ to $x = 2$ in the x direction, and from $y = 0$ to $y = 2$ in the y direction.
Now that the shape and domain of our pipe has been established, now we must discretize the domain into a set of finite points.
To simplify the code overall, the numerical grid will be regular: divide the rectangular domain by a set of equally spaced vertical lines and a set of equally spaced horizontal lines.
Let the number of vertical lines dividing the space be $nx$, and the number of horizontal lines be $ny$.
All of the variable names in this section correspond to the variables used in the python code below.
The spacing in between points in the x direction on the grid will equal the length of the domain in the x direction, divided by the number of lines dividing the space: $dx = \frac{2}{nx}$.
Similarily, the spacing in between points in the y direction on the grid will be $dy = \frac{2}{ny}$. 

\begin{lstlisting}

nx = 101
ny = 101

dx = 2.0/(nx-1)
dy = 2.0/(ny-1)

x = numpy.linspace(0,2,nx)
y = numpy.linspace(0,2,ny)

\end{lstlisting}

In the final two lines of code above, the variables $x$ and $y$ are each set equal to an array spanning from 0 to 2, with nx and ny entries respectively. Next, we should set the values for the density \textbf{rho}, dynamic viscosity \textbf{mu}, kinematic viscosity \textbf{nu}, and the length of timesteps in the simulation, \textbf{dt}.

\begin{lstlisting}

rho = 0.125
mu = 0.001
nu = mu/rho
dt = 0.001

\end{lstlisting}

The values of \textbf{rho}, \textbf{mu}, and \textbf{nu} are determined by the desired Reynolds number for the simulation, rather than by experimental values.
The Reynolds number is a dimensionless parameter that characterizes the type of flow, laminar, turbulent or something in between.
The Reynolds number takes into account the flow velocity, density, viscosity, and characteristic length and gives a general idea of what the flow should be like.

\begin{equation}
Re = \frac{\rho{v}{L}}{\mu}
\end{equation}

The flow through the pipe is driven by a pressure differential between the two ends of the pipe.
The pressure differential will be the variable \textbf{F}, and is constant throughout the simulation, so we will set that value now.

\begin{lstlisting}
F = 2
\end{lstlisting}

The stability of the program depends on the magnitude of the pressure differential.
The value of \textbf{F} can be thought of as a factor that scales up the velocity of the flow.
The resolution of the numerical grid must be increased as the pressure differential is increased.
Since the flow is driven by this pressure differential, the initial velocity throughout the pipe will be zero.

\begin{lstlisting}
u = numpy.zeros((ny,nx))
v = numpy.zeros((ny,nx))
p = numpy.zeros((ny,nx))
\end{lstlisting}

In the code above, \textbf{u} is a 2-dimensional array representing the x-component of the velocity vector field throughout the pipe, and \textbf{v} is a 2-dimensional array representing the y-component of the velocity.
Above, \textbf{p} is the 2-dimensional array representing the scalar field for pressure throughout the pipe, which is also initially zero.
One last variable we need to declare before running the code is \textbf{nt}, the number of time steps we would like the system to run for.
This is all of the code needed to setup the initial condition of the system.
